\title{Year 12 Chemistry Depth Study}
\author{
NESA \#: 34364338 
}

\documentclass[12pt, a4paper]{article}
\usepackage[margin=0.5in]{geometry}
\usepackage{parskip}
\usepackage{amsmath}
\usepackage[T1]{fontenc}

\usepackage{graphicx}
\graphicspath{ {./images/} }

\begin{document}
\maketitle

\begin{abstract}
This depth study report explores the role of equilibrium systems and reversible reactions within industrial applications, including the Contact Process and the Solvay Process.
\end{abstract}




\section{Contact Process}

The Contact Process is a multi-step industrial process used to produce sulfuric acid. 

\begin{enumerate}
	\item Produce sulfur dioxide from sulfur and excess oxygen
	\item Convert sulfur dioxide to sulfur trioxide
	\item Produce oleum (fuming sulfuric acid) from sulfur trioxide
	\item React oleum with water to produce concentrated sulfuric acid
\end{enumerate}

\subsection{Exploring the Contact Process}

\subsubsection{Producing sulfur dioxide}

In order to produce sulfur dioxide, an irreversible exothermic combustion reaction between sulfur and oxygen is used to produce sulfur dioxide.

\begin{align}
	S_{(s)} + O_{2}(g) \rightarrow SO_{2}(g)
\end{align}

\subsubsection{Sulfur dioxide to sulfur trioxide}

The conversion of sulfur dioxide to sulfur trioxide is an exothermic reversible reaction.

\begin{align}
	2SO_{2}(g) + O_{2} \leftrightharpoons 2SO_{3}(g)
\end{align}

\paragraph{Catalyst}
A catalyst of \textbf{vanadium(V) oxide} is used to reduce the activation energy, hence reducing the energy used for the reaction.

\pagebreak

\subsubsection{Producing concentrated sulfuric acid}

After the production of sulfur trioxide, sulfuric acid is produced. For a more stable reaction, sulfur trioxide is first dissolved in concentrated sulfuric acid, to produce \textbf{oleum} (or fuming sulfuric acid). Without this initial step, the reaction would produce sulfuric acid gas.

\begin{align}
	H_{2}SO_{4}(l) + SO_{3}(g) \rightarrow H_{2}S_{2}O_{7}(l)
\end{align}

Once oleum is produced, it is safely reacted with water to produce concentrated sulfuric acid. 

\begin{align}
	H_{2}S_{2}O_{7}(l) + H_{2}O(l) \rightarrow 2H_{2}SO_{4}(l)
\end{align}

As indicated by the molar ratio in reactions (3) and (4), the production of concentrated sulfuric acid from oleum produces twice as much concentrated sulfuric acid, as was originally used to produce oleum.

\subsection{The importance and uses of sulfuric acid}

Sulfuric acid is a key primary product used to produce a number of other chemical compounds, and the contact process allows them to be produced efficently in high concentrations at industrial scales. In particular, it is a key reactant for the production of phosphate-based fertilisers. For example, calcium phosphates are often used as fertilised and is produced by reacting phosphorite with sulfuric acid.

\begin{align}
	Ca_{3}(PO_{4})_{2} + H_{2}SO_{4} \rightarrow Ca(H_{2}PO_{4})_{2} + 2CaSO_{4}
\end{align}

Phosphate-based fertilisers are critical to the world's food supply, in ensuring that there is enough food production to sustain a growing world population. According to researchers, without phosphate and nitrogen-based fertilisers, humanity would only be able to produce half its current food production. \footnote{Faradji \& de Boer, 2016.} Although sulfur is relatively abundant, concentrated phosphorous is in short supply, which poses a threat to global food security.

Outside of fertiliser, sulfuric acid is also used as an industrial cleaning agent. It helps to remove oxidation and rust from iron and steel-based components in the passenger motor vehicles and major appliances industries, while also featuring as an ingredient in some household cleaning agents such as drain cleaners.

\subsection{Factors affecting equilibrium}

\textbf{Le Chatelier's Principle} states that the disturbance within a dynamic equilibrium from changing conditions, such as temperature, pressure and concentration, will result in an \textbf{equilibrium shift} to counteract and re-establish the equilibrium. In the case of the Contact Process, a shift right as far as possible would be desired such that as much product as possible is produced, leading to a more efficient process.

\subsubsection{Temperature}

The production of sulfur trioxide from sulfur trioxide from sulfur dioxide is an \textbf{exothermic} reaction.

By Le Chatelier's Principle, 

\subsubsection{Pressure}

\subsubsection{Concentration}

\pagebreak

\section{Bibliography}

British Broadcasting Corporation 2021, Sulfuric acid and the contact process, viewed 17 October 2021, \\ \textless{https://www.bbc.co.uk/bitesize/guides/zb7f3k7/revision/1}\textgreater

Clark, Jim 2021, \emph{The Contact Process}, Truro School in Cornwall, viewed 17 October 2021, \\ \textless{https://chem.libretexts.org/@go/page/3838}\textgreater

Department of Agriculture, Water and the Environment 2019, \emph{Sulfuric acid}, Commonwealth fo Australia, viewed 18 October 2021, \textless{http://www.npi.gov.au/resource/sulfuric-acid}\textgreater

Faradji, Charly & de Boer, Marissa, \emph{How the great phosphorus shortage could leave us all hungry}, The Conversation, viewed 18 October 2021, \\ \textless{https://theconversation.com/how-the-great-phosphorus-shortage-could-leave-us-all-hungry-54432}\textgreater

The Essential Chemistry Industry 2016, Sulfuric acid, viewed 18 October 2021, \\ \textless{https://www.essentialchemicalindustry.org/chemicals/sulfuric-acid.html}\textgreater

\end{document}

==================================================

inline notes:

- are we doing solvay process?
- manually use bibliography because the high school is too basic for bibtex :( - no Harvard AU support + they don't like [n]
- format date{today} later 
