\title{Year 12 Chemistry Depth Study}
\author{
NESA \#: 34364338 
}

\date{\today}

\documentclass[12pt, a4paper]{article}
\usepackage[margin=0.5in]{geometry}
\usepackage{parskip}
\usepackage{amsmath}
\usepackage[T1]{fontenc}

\usepackage{graphicx}
\graphicspath{ {./images/} }

\begin{document}
\maketitle

\begin{abstract}
This depth study report explores the role of equilibrium systems and reversible reactions within industrial applications, including the Contact Process and the Solvay Process.
\end{abstract}




\section{Contact Process}

The Contact Process is a three-step industrial process used to produce sulfuric acid. 

\begin{enumerate}
	\item Produce sulfur dioxide from sulfur and excess oxygen
	\item Convert sulfur dioxide to sulfur trioxide
	\item Dissolve sulfur trioxide in concentrated sulfuric acid to produce oleum (fuming sulfuric acid)
	\item React diluted sulfuric acid with water to produce twice as much sulfuric acid
\end{enumerate}

\subsection{Exploring the Contact Process}

\subsubsection{Producing sulfur dioxide}

In order to produce sulfur dioxide, a non-reversible exothermic combustion reaction betweeen sulfur and oxygen is used to produce sulfur dioxide.

\begin{align}
S_{(s)} + O_{2}(g) \rightarrow SO_{2}(g)
\end{align}

\subsubsection{Sulfur dioxide to sulfur trioxide}

The conversion of sulfur dioxide to sulfur drioxide is an exothermic reversible reacction.



\pagebreak

\section{Bibliography}
Clark, Jim 2021, \emph{The Contact Process}, Truro School in Cornwall, viewed 17 October 2021, \\ \textless{https://chem.libretexts.org/@go/page/3838}\textgreater

British Broadcasting Corporation 2021, Sufuric acid and the contact process, viewed 17 October 2021, \\ \textless{https://www.bbc.co.uk/bitesize/guides/zb7f3k7/revision/1}\textgreater

\end{document}

==================================================

inline notes:

- are we doing solvay process?
- manually use bibliography because the high school is too basic for bibtex :( - no Harvard AU support + they don't like [n]
